\documentclass[ignorenonframetext,]{beamer}
\setbeamertemplate{caption}[numbered]
\setbeamertemplate{caption label separator}{: }
\setbeamercolor{caption name}{fg=normal text.fg}
\beamertemplatenavigationsymbolsempty
\usepackage{lmodern}
\usepackage{amssymb,amsmath}
\usepackage{ifxetex,ifluatex}
\usepackage{fixltx2e} % provides \textsubscript
\ifnum 0\ifxetex 1\fi\ifluatex 1\fi=0 % if pdftex
  \usepackage[T1]{fontenc}
  \usepackage[utf8]{inputenc}
\else % if luatex or xelatex
  \ifxetex
    \usepackage{mathspec}
  \else
    \usepackage{fontspec}
  \fi
  \defaultfontfeatures{Ligatures=TeX,Scale=MatchLowercase}
\fi
\usecolortheme{albatross}
% use upquote if available, for straight quotes in verbatim environments
\IfFileExists{upquote.sty}{\usepackage{upquote}}{}
% use microtype if available
\IfFileExists{microtype.sty}{%
\usepackage{microtype}
\UseMicrotypeSet[protrusion]{basicmath} % disable protrusion for tt fonts
}{}
\newif\ifbibliography

%% highlighting-macros %%
\usepackage{color} 
\usepackage{fancyvrb}
\newcommand{\VerbBar}{|}
\newcommand{\VERB}{\Verb[commandchars=\\\{\}]}
\DefineVerbatimEnvironment{Highlighting}{Verbatim}{commandchars=\\\{\}}
% Add ',fontsize=\small' for more characters per line
\usepackage{framed}

% change shadecolor to match with albatross Beamer theme
\colorlet{beamerbgcolor}{blue!50!black}
\definecolor{shadecolor}{named}{beamerbgcolor}
%\definecolor{shadecolor}{RGB}{248,248,248} % old

\definecolor{numbercolor}{rgb}{1.0, 0.75, 0.0}

\newenvironment{Shaded}{\begin{snugshade}}{\end{snugshade}}
% 
\newcommand{\KeywordTok}[1]{\textcolor{white}{\textbf{{#1}}}}
\newcommand{\NormalTok}[1]{\textcolor{yellow}{{#1}}}
\newcommand{\DataTypeTok}[1]{\textcolor{yellow}{{#1}}}
\newcommand{\DecValTok}[1]{\textcolor{numbercolor}{{#1}}}
\newcommand{\BaseNTok}[1]{\textcolor{numbercolor}{{#1}}}
\newcommand{\FloatTok}[1]{\textcolor{numbercolor}{{#1}}}
\newcommand{\ConstantTok}[1]{\textcolor{numbercolor}{{#1}}}
\newcommand{\StringTok}[1]{\textcolor{yellow}{{#1}}}

\newcommand{\CharTok}[1]{\textcolor[rgb]{0.31,0.60,0.02}{{#1}}}
\newcommand{\SpecialCharTok}[1]{\textcolor[rgb]{0.00,0.00,0.00}{{#1}}}
\newcommand{\VerbatimStringTok}[1]{\textcolor[rgb]{0.31,0.60,0.02}{{#1}}}
\newcommand{\SpecialStringTok}[1]{\textcolor[rgb]{0.31,0.60,0.02}{{#1}}}
\newcommand{\ImportTok}[1]{{#1}}
\newcommand{\CommentTok}[1]{\textcolor[rgb]{0.56,0.35,0.01}{\textit{{#1}}}}
\newcommand{\DocumentationTok}[1]{\textcolor[rgb]{0.56,0.35,0.01}{\textbf{\textit{{#1}}}}}
\newcommand{\AnnotationTok}[1]{\textcolor[rgb]{0.56,0.35,0.01}{\textbf{\textit{{#1}}}}}
\newcommand{\CommentVarTok}[1]{\textcolor[rgb]{0.56,0.35,0.01}{\textbf{\textit{{#1}}}}}
\newcommand{\OtherTok}[1]{\textcolor[rgb]{0.56,0.35,0.01}{{#1}}}
\newcommand{\FunctionTok}[1]{\textcolor[rgb]{0.00,0.00,0.00}{{#1}}}
\newcommand{\VariableTok}[1]{\textcolor[rgb]{0.00,0.00,0.00}{{#1}}}
\newcommand{\ControlFlowTok}[1]{\textcolor[rgb]{0.13,0.29,0.53}{\textbf{{#1}}}}
\newcommand{\OperatorTok}[1]{\textcolor[rgb]{0.81,0.36,0.00}{\textbf{{#1}}}}
\newcommand{\BuiltInTok}[1]{{#1}}
\newcommand{\ExtensionTok}[1]{{#1}}
\newcommand{\PreprocessorTok}[1]{\textcolor[rgb]{0.56,0.35,0.01}{\textit{{#1}}}}
\newcommand{\AttributeTok}[1]{\textcolor[rgb]{0.77,0.63,0.00}{{#1}}}
\newcommand{\RegionMarkerTok}[1]{{#1}}
\newcommand{\InformationTok}[1]{\textcolor[rgb]{0.56,0.35,0.01}{\textbf{\textit{{#1}}}}}
\newcommand{\WarningTok}[1]{\textcolor[rgb]{0.56,0.35,0.01}{\textbf{\textit{{#1}}}}}
\newcommand{\AlertTok}[1]{\textcolor[rgb]{0.94,0.16,0.16}{{#1}}}
\newcommand{\ErrorTok}[1]{\textcolor[rgb]{0.64,0.00,0.00}{\textbf{{#1}}}}

%% End of highlighting-macros %%


% Prevent slide breaks in the middle of a paragraph:
\widowpenalties 1 10000
\raggedbottom

\AtBeginPart{
  \let\insertpartnumber\relax
  \let\partname\relax
  \frame{\partpage}
}
\AtBeginSection{
  \ifbibliography
  \else
    \let\insertsectionnumber\relax
    \let\sectionname\relax
    \frame{\sectionpage}
  \fi
}
\AtBeginSubsection{
  \let\insertsubsectionnumber\relax
  \let\subsectionname\relax
  \frame{\subsectionpage}
}

\setlength{\emergencystretch}{3em}  % prevent overfull lines
\providecommand{\tightlist}{%
  \setlength{\itemsep}{0pt}\setlength{\parskip}{0pt}}
\setcounter{secnumdepth}{0}

\title{Lecture 7 Graphical Data Analysis}
\author{The Great Courses}
\date{}

\begin{document}
\frame{\titlepage}

\begin{frame}{Today's topic: Graphical Data analysis}

Statistical graphs are useful in helping us visualize data. Through
graphs, we:

\begin{itemize}
\tightlist
\item
  Understand data properties
\item
  Find patterns in data
\item
  Suggest modeling strategies
\item
  ``Debug'' our analyses
\item
  Communicate results
\end{itemize}

\end{frame}

\begin{frame}{Learning Objectives for today}

\begin{itemize}
\item
  Define and identify basic numerical and graphical summeries of data
\item
  Use R for calculating descriptive statistics, making graphs, and
  writing functions
\end{itemize}

\end{frame}

\begin{frame}{Iris Data}

The Iris dataset is widely used throughout statistical science for
illustrating various problems in statistical graphics, multivariate
statistics and machine learning.

\begin{itemize}
\item
  It's a small, but non-trivial dataset.
\item
  The data values are real (as opposed to simulated) and are of high
  quality (collected with minimal error).
\item
  The data were used by the celebrated British statistician Ronald
  Fisher in 1936. (Later he was knighted and became Sir Ronald.)
\item
  Using a few famous datasets is one of the traditions we hand down in
  statistics! (Also, when comparing old and new methods, or in
  evaluating any method, it's helpful to try them out on known datasets,
  thus maintaining continuity in how we assess methods.)
\end{itemize}

\end{frame}

\begin{frame}{Iris Data}

The Iris dataset is most commonly used for on pattern recognition in
statistics. The dataset contains 3 classes of 50 instances each, where
each class refers to a type of iris plant, with the following
attributes:

\begin{itemize}
\tightlist
\item
  Sepal Length
\item
  Sepal Width
\item
  Petal Length\\
\item
  Petal Width
\item
  class: Iris setosa, Iris versicolor, Iris virginica
\end{itemize}

\end{frame}

\begin{frame}[fragile]{Load Data}

The Iris data is in the \texttt{datasets} library in R. Type the
following commands:

\small

\begin{Shaded}
\begin{Highlighting}[]
\KeywordTok{library}\NormalTok{(datasets)}
\KeywordTok{library}\NormalTok{(RColorBrewer)}
\KeywordTok{attach}\NormalTok{(iris)}
\KeywordTok{head}\NormalTok{(iris)}
  \NormalTok{Sepal.Length Sepal.Width Petal.Length Petal.Width Species}
\DecValTok{1}          \FloatTok{5.1}         \FloatTok{3.5}          \FloatTok{1.4}         \FloatTok{0.2}  \NormalTok{setosa}
\DecValTok{2}          \FloatTok{4.9}         \FloatTok{3.0}          \FloatTok{1.4}         \FloatTok{0.2}  \NormalTok{setosa}
\DecValTok{3}          \FloatTok{4.7}         \FloatTok{3.2}          \FloatTok{1.3}         \FloatTok{0.2}  \NormalTok{setosa}
\DecValTok{4}          \FloatTok{4.6}         \FloatTok{3.1}          \FloatTok{1.5}         \FloatTok{0.2}  \NormalTok{setosa}
\DecValTok{5}          \FloatTok{5.0}         \FloatTok{3.6}          \FloatTok{1.4}         \FloatTok{0.2}  \NormalTok{setosa}
\DecValTok{6}          \FloatTok{5.4}         \FloatTok{3.9}          \FloatTok{1.7}         \FloatTok{0.4}  \NormalTok{setosa}
\end{Highlighting}
\end{Shaded}

\end{frame}

\begin{frame}{Bar Plots}

Let's begin our analysis.

\begin{itemize}
\item
  \textbf{Bar Plots} are useful for showing comparisons across several
  groups. Although it looks like a histogram, a bar plot is plotted over
  a label that represents a category (e.g., Iris type).
\item
  One indication of the difference between a bar plot and histogram:
  It's always appropriate to talk about the skewness of a histogram;
  that is, the tendency of the observations to fall more on the low end
  or the high end of the X axis.
\item
  However, on bar plots, the X axis can sometimes be categorical -
  (i.e.~not quantitative.)
\end{itemize}

\end{frame}

\begin{frame}[fragile]{Bar Plots}

\small

\begin{Shaded}
\begin{Highlighting}[]
\KeywordTok{barplot}\NormalTok{(iris$Petal.Length, }\DataTypeTok{main =} \StringTok{"Petal Length"}\NormalTok{) }
\end{Highlighting}
\end{Shaded}

\begin{center}\includegraphics[width=1\linewidth]{L7_Slides-new_files/figure-beamer/unnamed-chunk-2-1} \end{center}

\end{frame}

\begin{frame}[fragile]{Bar Plot}

\small

\begin{Shaded}
\begin{Highlighting}[]
\KeywordTok{barplot}\NormalTok{(iris$Sepal.Length, }
        \DataTypeTok{col=} \KeywordTok{brewer.pal}\NormalTok{(}\DecValTok{3}\NormalTok{,}\StringTok{"Set1"}\NormalTok{), }\DataTypeTok{main =} \StringTok{"Sepal Length"}\NormalTok{)}
\end{Highlighting}
\end{Shaded}

\begin{center}\includegraphics[width=2.5\linewidth]{L7_Slides-new_files/figure-beamer/unnamed-chunk-3-1} \end{center}

\end{frame}

\begin{frame}[fragile]{Bar Plot}

\small

\begin{Shaded}
\begin{Highlighting}[]
\KeywordTok{barplot}\NormalTok{(}\KeywordTok{table}\NormalTok{(iris$Species,iris$Sepal.Length), }
        \DataTypeTok{col  =} \KeywordTok{brewer.pal}\NormalTok{(}\DecValTok{3}\NormalTok{,}\StringTok{"Set1"}\NormalTok{), }
        \DataTypeTok{main =} \StringTok{"Stacked Plot of Sepal Length by Species"}\NormalTok{)}
\end{Highlighting}
\end{Shaded}

\begin{center}\includegraphics[width=1\linewidth]{L7_Slides-new_files/figure-beamer/unnamed-chunk-4-1} \end{center}

\end{frame}

\begin{frame}[fragile]{Summary Statistics}

The \texttt{summary} function is a quick and easy way to assess the
statistical properties of each attribute. These values are displayed
graphically in a box plot.

\small

\begin{Shaded}
\begin{Highlighting}[]
\KeywordTok{summary}\NormalTok{(iris)}
  \NormalTok{Sepal.Length    Sepal.Width     Petal.Length    Petal.Width   }
 \NormalTok{Min.   :}\FloatTok{4.300}   \NormalTok{Min.   :}\FloatTok{2.000}   \NormalTok{Min.   :}\FloatTok{1.000}   \NormalTok{Min.   :}\FloatTok{0.100}  
 \NormalTok{1st Qu.:}\FloatTok{5.100}   \NormalTok{1st Qu.:}\FloatTok{2.800}   \NormalTok{1st Qu.:}\FloatTok{1.600}   \NormalTok{1st Qu.:}\FloatTok{0.300}  
 \NormalTok{Median :}\FloatTok{5.800}   \NormalTok{Median :}\FloatTok{3.000}   \NormalTok{Median :}\FloatTok{4.350}   \NormalTok{Median :}\FloatTok{1.300}  
 \NormalTok{Mean   :}\FloatTok{5.843}   \NormalTok{Mean   :}\FloatTok{3.057}   \NormalTok{Mean   :}\FloatTok{3.758}   \NormalTok{Mean   :}\FloatTok{1.199}  
 \NormalTok{3rd Qu.:}\FloatTok{6.400}   \NormalTok{3rd Qu.:}\FloatTok{3.300}   \NormalTok{3rd Qu.:}\FloatTok{5.100}   \NormalTok{3rd Qu.:}\FloatTok{1.800}  
 \NormalTok{Max.   :}\FloatTok{7.900}   \NormalTok{Max.   :}\FloatTok{4.400}   \NormalTok{Max.   :}\FloatTok{6.900}   \NormalTok{Max.   :}\FloatTok{2.500}  
       \NormalTok{Species  }
 \NormalTok{setosa    :}\DecValTok{50}  
 \NormalTok{versicolor:}\DecValTok{50}  
 \NormalTok{virginica :}\DecValTok{50}  
                
                
                
\end{Highlighting}
\end{Shaded}

\end{frame}

\begin{frame}[fragile]{Box Plots}

\begin{itemize}
\item
  Box plots are used to compactly show many pieces of information about
  a variables distribution and is useful for visualizing the spread of
  the data.
\item
  Box plots show \textbf{five statistically important numbers} - the
  minimum, the 25th percentile, the median, the 75th percentile and the
  maximum.
\end{itemize}

\small

\begin{Shaded}
\begin{Highlighting}[]
\KeywordTok{boxplot}\NormalTok{(iris$Sepal.Length, }\DataTypeTok{main =} \StringTok{"Sepal Length"}\NormalTok{)}
\end{Highlighting}
\end{Shaded}

\begin{center}\includegraphics[width=0.8\linewidth]{L7_Slides-new_files/figure-beamer/unnamed-chunk-6-1} \end{center}

\vfill

\end{frame}

\begin{frame}[fragile]{Box Plots}

\small

\begin{Shaded}
\begin{Highlighting}[]
\KeywordTok{boxplot}\NormalTok{(iris[,}\DecValTok{1}\NormalTok{:}\DecValTok{4}\NormalTok{], }
        \DataTypeTok{names=}\KeywordTok{c}\NormalTok{(}\StringTok{"Sep L"}\NormalTok{, }\StringTok{"Sep W"}\NormalTok{, }\StringTok{"Pet L"}\NormalTok{, }\StringTok{"Pet W"}\NormalTok{))}
\end{Highlighting}
\end{Shaded}

\begin{center}\includegraphics[width=2\linewidth]{L7_Slides-new_files/figure-beamer/boxplot1-1} \end{center}

\end{frame}

\begin{frame}[fragile]{Box Plots}

A box plot can also be used to show how one attribute
\texttt{petal.length} varies with another attribute \texttt{iris.type}.

\small

\begin{Shaded}
\begin{Highlighting}[]
\KeywordTok{boxplot}\NormalTok{(iris$Petal.Length~iris$Species, }
        \DataTypeTok{main =} \StringTok{"Petal Length vs. Species"}\NormalTok{) }
\end{Highlighting}
\end{Shaded}

\begin{center}\includegraphics[width=1\linewidth]{L7_Slides-new_files/figure-beamer/unnamed-chunk-7-1} \end{center}

\end{frame}

\begin{frame}[fragile]{Box Plot}

We can visualize how the spread of \texttt{Sepal\ Length} changes across
various categories of \texttt{Species}. A color palette is a group of
colors that is used to make the graph more appealing and help create
visual distinctions in the data.

\end{frame}

\begin{frame}[fragile]{Box Plot}

\small

\begin{Shaded}
\begin{Highlighting}[]
\KeywordTok{boxplot}\NormalTok{(iris$Sepal.Length~iris$Species,}
        \DataTypeTok{col=}\KeywordTok{heat.colors}\NormalTok{(}\DecValTok{3}\NormalTok{), }
        \DataTypeTok{main =} \StringTok{"Sepal Length vs. Species"}\NormalTok{)}
\end{Highlighting}
\end{Shaded}

\begin{center}\includegraphics{L7_Slides-new_files/figure-beamer/unnamed-chunk-8-1} \end{center}

\end{frame}

\begin{frame}[fragile]{Scatter Plot}

Scatter plots help in visualizing data easily and for simple data
inspection. Try the following code.

\small

\begin{Shaded}
\begin{Highlighting}[]
\KeywordTok{plot}\NormalTok{(iris$Petal.Length, }\DataTypeTok{main=}\StringTok{"Petal Length"}\NormalTok{, }
     \DataTypeTok{ylab =} \StringTok{"Petal Length"}\NormalTok{, }\DataTypeTok{xlab =} \StringTok{"Species"}\NormalTok{) }
\end{Highlighting}
\end{Shaded}

\begin{center}\includegraphics{L7_Slides-new_files/figure-beamer/unnamed-chunk-9-1} \end{center}

\end{frame}

\begin{frame}[fragile]{Scatter Plot}

Let's generate corresponding scatterplots for Petal.Width, Sepal.Length
and Sepal.Width.

\small

\begin{Shaded}
\begin{Highlighting}[]
\KeywordTok{par}\NormalTok{(}\DataTypeTok{mfrow=}\KeywordTok{c}\NormalTok{(}\DecValTok{1}\NormalTok{,}\DecValTok{2}\NormalTok{))}
\KeywordTok{plot}\NormalTok{(iris$Petal.Length, }\DataTypeTok{main=}\StringTok{"Petal Length"}\NormalTok{) }
\KeywordTok{plot}\NormalTok{(iris$Sepal.Length, }\DataTypeTok{main=}\StringTok{"Sepal Length"}\NormalTok{)}
\end{Highlighting}
\end{Shaded}

\begin{center}\includegraphics{L7_Slides-new_files/figure-beamer/unnamed-chunk-10-1} \end{center}

\end{frame}

\begin{frame}[fragile]{Scatter Plot}

\small

\begin{Shaded}
\begin{Highlighting}[]
\KeywordTok{par}\NormalTok{(}\DataTypeTok{mfrow=}\KeywordTok{c}\NormalTok{(}\DecValTok{1}\NormalTok{,}\DecValTok{2}\NormalTok{))}
\KeywordTok{plot}\NormalTok{(iris$Petal.Width, }\DataTypeTok{main=}\StringTok{"Petal Width"}\NormalTok{) }
\KeywordTok{plot}\NormalTok{(iris$Sepal.Width, }\DataTypeTok{main=}\StringTok{"Sepal Width"}\NormalTok{) }
\end{Highlighting}
\end{Shaded}

\begin{center}\includegraphics{L7_Slides-new_files/figure-beamer/unnamed-chunk-11-1} \end{center}

What are our observations? Which plots help us distinguish between
groups?

\end{frame}

\begin{frame}{Scatter Plots}

\begin{itemize}
\item
  Scatter plots are used to plot two variables against each other. We
  can add a third dimension by coloring the data values according to
  their Species.
\item
  For datasets with only a few attributes, we can construct and view all
  the pairwise scatter plots.
\end{itemize}

\end{frame}

\begin{frame}[fragile]{Pairwise Plots}

\small

\begin{Shaded}
\begin{Highlighting}[]
\KeywordTok{pairs}\NormalTok{(}\KeywordTok{as.matrix}\NormalTok{(iris[,-}\DecValTok{5}\NormalTok{]), }\DataTypeTok{pch=}\DecValTok{21}\NormalTok{, }\DataTypeTok{bg=}\KeywordTok{c}\NormalTok{(}\StringTok{"red"}\NormalTok{, }\StringTok{"blue"}\NormalTok{, }\StringTok{"green"}\NormalTok{)[}\KeywordTok{unclass}\NormalTok{(iris$Species)])}
\end{Highlighting}
\end{Shaded}

\end{frame}

\begin{frame}{Pairwise Plots}

\small

\begin{center}\includegraphics[width=1\linewidth]{L7_Slides-new_files/figure-beamer/unnamed-chunk-13-1} \end{center}

\end{frame}

\begin{frame}{Pairwise Plots}

\begin{itemize}
\item
  Since the upper and lower graphs are duplicates of each other, we can
  augment our code to display the \textbf{correlation} between our
  variables in the upper level.
\item
  The correlation measures the strength of the relationship between two
  random variables.
\item
  Correlations range from -1 to 1, where:
\end{itemize}

-- Values near 1 indicate a strong positive relationship

-- Values near -1 indicate a strong negative relationship

-- Values near 0 indicate no relationship.

\end{frame}

\begin{frame}[fragile]{Pairwise Plots}

\small

\begin{Shaded}
\begin{Highlighting}[]
\NormalTok{panel.pearson <-}\StringTok{ }\NormalTok{function(x, y, ...) \{}
  \NormalTok{horizontal <-}\StringTok{ }\NormalTok{(}\KeywordTok{par}\NormalTok{(}\StringTok{"usr"}\NormalTok{)[}\DecValTok{1}\NormalTok{] +}\StringTok{ }\KeywordTok{par}\NormalTok{(}\StringTok{"usr"}\NormalTok{)[}\DecValTok{2}\NormalTok{]) /}\StringTok{ }\DecValTok{2}\NormalTok{;}
  \NormalTok{vertical <-}\StringTok{ }\NormalTok{(}\KeywordTok{par}\NormalTok{(}\StringTok{"usr"}\NormalTok{)[}\DecValTok{3}\NormalTok{] +}\StringTok{ }\KeywordTok{par}\NormalTok{(}\StringTok{"usr"}\NormalTok{)[}\DecValTok{4}\NormalTok{]) /}\StringTok{ }\DecValTok{2}\NormalTok{;}
  \KeywordTok{text}\NormalTok{(horizontal, vertical, }\KeywordTok{format}\NormalTok{(}\KeywordTok{abs}\NormalTok{(}\KeywordTok{cor}\NormalTok{(x,y)), }\DataTypeTok{digits=}\DecValTok{2}\NormalTok{))\}}

\KeywordTok{pairs}\NormalTok{(}\KeywordTok{as.matrix}\NormalTok{(iris[}\DecValTok{1}\NormalTok{:}\DecValTok{4}\NormalTok{]), }\DataTypeTok{main =} \StringTok{"Iris Data"}\NormalTok{, }\DataTypeTok{pch =} \DecValTok{21}\NormalTok{, }
      \DataTypeTok{bg =} \KeywordTok{c}\NormalTok{(}\StringTok{"red"}\NormalTok{,}\StringTok{"blue"}\NormalTok{, }\StringTok{"green"}\NormalTok{)[}\KeywordTok{unclass}\NormalTok{(iris$Species)],}
      \DataTypeTok{upper.panel=}\NormalTok{panel.pearson)}
\end{Highlighting}
\end{Shaded}

\end{frame}

\begin{frame}{Pairwise Plots}

\small

\begin{center}\includegraphics[width=1\linewidth]{L7_Slides-new_files/figure-beamer/unnamed-chunk-15-1} \end{center}

\end{frame}

\begin{frame}{Histograms}

\begin{itemize}
\item
  A \textbf{Histogram} is a plot that breaks the data into bins (or
  breaks) and shows the frequency distribution of those bins.
\item
  We can change the breaks to see the effect it has data visualization.
\end{itemize}

\end{frame}

\begin{frame}[fragile]{Histograms}

Let's create some histograms of our Iris data. The number of bins in the
histogram is variable.

\small

\begin{Shaded}
\begin{Highlighting}[]
\KeywordTok{hist}\NormalTok{(iris$Petal.Width, }\DataTypeTok{breaks=}\DecValTok{13}\NormalTok{)}
\end{Highlighting}
\end{Shaded}

\begin{center}\includegraphics{L7_Slides-new_files/figure-beamer/unnamed-chunk-16-1} \end{center}

\end{frame}

\begin{frame}[fragile]{Histograms}

\small

\begin{Shaded}
\begin{Highlighting}[]
\KeywordTok{hist}\NormalTok{(iris$Petal.Width, }\DataTypeTok{breaks=}\DecValTok{25}\NormalTok{)}
\end{Highlighting}
\end{Shaded}

\begin{center}\includegraphics{L7_Slides-new_files/figure-beamer/unnamed-chunk-17-1} \end{center}

\end{frame}

\begin{frame}[fragile]{Histograms}

We can create custom break points by defining a sequence vector,
\texttt{b}, that ranges from \texttt{min(iris\$Petal.Width)} to the
\texttt{max(iris\$Petal.Width)} with a specified number of breaks.

\small

\begin{Shaded}
\begin{Highlighting}[]
\NormalTok{b <-}\StringTok{ }\KeywordTok{seq}\NormalTok{(}\KeywordTok{min}\NormalTok{(iris$Petal.Width), }
         \KeywordTok{max}\NormalTok{(iris$Petal.Width), }\DataTypeTok{length=}\DecValTok{11}\NormalTok{)}
\NormalTok{b}
\KeywordTok{hist}\NormalTok{(iris$Petal.Width, }\DataTypeTok{breaks=}\NormalTok{b, }
     \DataTypeTok{xlab=}\StringTok{"Petal Width"}\NormalTok{, }\DataTypeTok{main=}\StringTok{"Histogram of Petal Width"}\NormalTok{)}
\end{Highlighting}
\end{Shaded}

\end{frame}

\begin{frame}{Histograms}

\small

\begin{center}\includegraphics[width=0.65\linewidth]{L7_Slides-new_files/figure-beamer/unnamed-chunk-19-1} \end{center}

\end{frame}

\begin{frame}[fragile]{Density Plots}

Density plots can be viewed as smoothed versions of a histogram. We can
estimate the density using R's \texttt{density} function

\small

\begin{Shaded}
\begin{Highlighting}[]
\NormalTok{dens.pw =}\StringTok{ }\KeywordTok{density}\NormalTok{(iris$Petal.Width)}
\KeywordTok{plot}\NormalTok{(dens.pw, }\DataTypeTok{ylab =} \StringTok{"Frequency"}\NormalTok{, }\DataTypeTok{xlab =} \StringTok{"Width"}\NormalTok{, }
     \DataTypeTok{main=} \StringTok{"Petal Width Density"}\NormalTok{)}
\end{Highlighting}
\end{Shaded}

\begin{center}\includegraphics{L7_Slides-new_files/figure-beamer/unnamed-chunk-20-1} \end{center}

\end{frame}

\begin{frame}[fragile]{Density Plots}

\small

\begin{Shaded}
\begin{Highlighting}[]
\NormalTok{dens.pl =}\StringTok{ }\KeywordTok{density}\NormalTok{(iris$Petal.Length)}
\KeywordTok{plot}\NormalTok{(dens.pl, }\DataTypeTok{ylab =} \StringTok{"Frequency"}\NormalTok{, }\DataTypeTok{xlab =} \StringTok{"Length"}\NormalTok{, }
     \DataTypeTok{main=} \StringTok{"Petal Length Density"}\NormalTok{)}
\end{Highlighting}
\end{Shaded}

\begin{center}\includegraphics{L7_Slides-new_files/figure-beamer/unnamed-chunk-21-1} \end{center}

\end{frame}

\begin{frame}[fragile]{Density Plots}

\small

\begin{Shaded}
\begin{Highlighting}[]
\NormalTok{dens.sw =}\StringTok{ }\KeywordTok{density}\NormalTok{(iris$Sepal.Width)}
\KeywordTok{plot}\NormalTok{(dens.sw, }\DataTypeTok{ylab =} \StringTok{"Frequency"}\NormalTok{, }\DataTypeTok{xlab =} \StringTok{"Width"}\NormalTok{, }
     \DataTypeTok{main=} \StringTok{"Sepal Width Density"}\NormalTok{)}
\end{Highlighting}
\end{Shaded}

\begin{center}\includegraphics{L7_Slides-new_files/figure-beamer/unnamed-chunk-22-1} \end{center}

\end{frame}

\begin{frame}[fragile]{Density Plots}

\small

\begin{Shaded}
\begin{Highlighting}[]
\NormalTok{dens.sl =}\StringTok{ }\KeywordTok{density}\NormalTok{(iris$Sepal.Length)}
\KeywordTok{plot}\NormalTok{(dens.sl, }\DataTypeTok{ylab =} \StringTok{"Frequency"}\NormalTok{, }\DataTypeTok{xlab =} \StringTok{"Length"}\NormalTok{, }
     \DataTypeTok{main=} \StringTok{"Sepal Length Density"}\NormalTok{)}
\end{Highlighting}
\end{Shaded}

\begin{center}\includegraphics{L7_Slides-new_files/figure-beamer/unnamed-chunk-23-1} \end{center}

\end{frame}

\begin{frame}[fragile]{Density Plots}

Let's also look at the density function of \texttt{Petal.Length} for
each of the three classes of irises.

\small

\small

\end{frame}

\begin{frame}{Density Plots}

\small

\begin{center}\includegraphics[width=1\linewidth]{L7_Slides-new_files/figure-beamer/unnamed-chunk-26-1} \end{center}

\end{frame}

\begin{frame}[fragile]{Quantile Plots}

We can calculate the quantiles of the iris dataset to compare them to
those of a normal distribution.

\small

\small

\begin{Shaded}
\begin{Highlighting}[]
\KeywordTok{qqnorm}\NormalTok{(quantile.virginica, }\DataTypeTok{main=}\StringTok{"Virginica"}\NormalTok{)}
\KeywordTok{qqline}\NormalTok{(quantile.virginica)}
\end{Highlighting}
\end{Shaded}

\begin{center}\includegraphics{L7_Slides-new_files/figure-beamer/unnamed-chunk-28-1} \end{center}

\end{frame}

\begin{frame}[fragile]{Quantile Plots}

\small

\begin{Shaded}
\begin{Highlighting}[]
\KeywordTok{qqnorm}\NormalTok{(quantile.versicolor, }\DataTypeTok{main=}\StringTok{"Versicolor"}\NormalTok{)}
\KeywordTok{qqline}\NormalTok{(quantile.versicolor)}
\end{Highlighting}
\end{Shaded}

\begin{center}\includegraphics{L7_Slides-new_files/figure-beamer/unnamed-chunk-29-1} \end{center}

\end{frame}

\begin{frame}[fragile]{Quantile Plots}

\small

\begin{Shaded}
\begin{Highlighting}[]
\KeywordTok{qqnorm}\NormalTok{(quantile.setosa, }\DataTypeTok{main=}\StringTok{"Setosa"}\NormalTok{)}
\KeywordTok{qqline}\NormalTok{(quantile.setosa)}
\end{Highlighting}
\end{Shaded}

\begin{center}\includegraphics{L7_Slides-new_files/figure-beamer/unnamed-chunk-30-1} \end{center}

\end{frame}

\begin{frame}{Contour Plots}

Density estimation is available for higher dimensional data using
Contour plots.

\begin{itemize}
\item
  A \textbf{contour plot} is a graph that explores the potential
  relationship among three variables.
\item
  Contour plots display the 3-dimensional relationship in two
  dimensions, with x and y variables plotted on the x and y scales and
  the z variable represented by contours.
\item
  A contour plot is like a topographical map in which x, y, and z values
  are plotted instead of longitude, latitude, and elevation.
\end{itemize}

\end{frame}

\begin{frame}[fragile]{Contour Plots}

\small

\begin{Shaded}
\begin{Highlighting}[]
\KeywordTok{library}\NormalTok{(MASS)}
\NormalTok{petal.dens =}\StringTok{ }\KeywordTok{kde2d}\NormalTok{(iris$Petal.Length, iris$Petal.Width)}
\KeywordTok{contour}\NormalTok{(petal.dens)}
\end{Highlighting}
\end{Shaded}

\begin{center}\includegraphics{L7_Slides-new_files/figure-beamer/unnamed-chunk-31-1} \end{center}

\end{frame}

\begin{frame}[fragile]{Contour Plots}

The plot may also be viewed as a heatmap, with brighter colors denoting
higher values. \small

\begin{Shaded}
\begin{Highlighting}[]
\KeywordTok{image}\NormalTok{(petal.dens)}
\end{Highlighting}
\end{Shaded}

\begin{center}\includegraphics{L7_Slides-new_files/figure-beamer/unnamed-chunk-32-1} \end{center}

\end{frame}

\end{document}
